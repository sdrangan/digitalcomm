\documentclass[11pt]{article}

\usepackage{fullpage}
\usepackage{amsmath, amssymb, bm, cite, epsfig, psfrag}
\usepackage{graphicx}
\usepackage{float}
\usepackage{amsthm}
\usepackage{amsfonts}
\usepackage{listings}
\usepackage{cite}
\usepackage{hyperref}
\usepackage{tikz}
\usepackage{enumerate}
\usepackage[outercaption]{sidecap}
\usetikzlibrary{shapes,arrows}
%\usetikzlibrary{dsp,chains}

%\restylefloat{figure}
%\theoremstyle{plain}      \newtheorem{theorem}{Theorem}
%\theoremstyle{definition} \newtheorem{definition}{Definition}

\def\del{\partial}
\def\ds{\displaystyle}
\def\ts{\textstyle}
\def\beq{\begin{equation}}
\def\eeq{\end{equation}}
\def\beqa{\begin{eqnarray}}
\def\eeqa{\end{eqnarray}}
\def\beqan{\begin{eqnarray*}}
\def\eeqan{\end{eqnarray*}}
\def\nn{\nonumber}
\def\binomial{\mathop{\mathrm{binomial}}}
\def\half{{\ts\frac{1}{2}}}
\def\Half{{\frac{1}{2}}}
\def\N{{\mathbb{N}}}
\def\Z{{\mathbb{Z}}}
\def\Q{{\mathbb{Q}}}
\def\F{{\mathbb{F}}}
\def\R{{\mathbb{R}}}
\def\C{{\mathbb{C}}}
\def\argmin{\mathop{\mathrm{arg\,min}}}
\def\argmax{\mathop{\mathrm{arg\,max}}}
%\def\span{\mathop{\mathrm{span}}}
\def\diag{\mathop{\mathrm{diag}}}
\def\x{\times}
\def\limn{\lim_{n \rightarrow \infty}}
\def\liminfn{\liminf_{n \rightarrow \infty}}
\def\limsupn{\limsup_{n \rightarrow \infty}}
\def\GV{Guo and Verd{\'u}}
\def\MID{\,|\,}
\def\MIDD{\,;\,}

\newtheorem{proposition}{Proposition}
\newtheorem{definition}{Definition}
\newtheorem{theorem}{Theorem}
\newtheorem{lemma}{Lemma}
\newtheorem{corollary}{Corollary}
\newtheorem{assumption}{Assumption}
\newtheorem{claim}{Claim}
\def\qed{\mbox{} \hfill $\Box$}
\setlength{\unitlength}{1mm}

\def\bhat{\widehat{b}}
\def\ehat{\widehat{e}}
\def\phat{\widehat{p}}
\def\qhat{\widehat{q}}
\def\rhat{\widehat{r}}
\def\shat{\widehat{s}}
\def\uhat{\widehat{u}}
\def\ubar{\overline{u}}
\def\vhat{\widehat{v}}
\def\xhat{\widehat{x}}
\def\xbar{\overline{x}}
\def\zhat{\widehat{z}}
\def\zbar{\overline{z}}
\def\la{\leftarrow}
\def\ra{\rightarrow}
\def\MSE{\mbox{\small \sffamily MSE}}
\def\SNR{\mbox{\small \sffamily SNR}}
\def\SINR{\mbox{\small \sffamily SINR}}
\def\arr{\rightarrow}
\def\Exp{\mathbb{E}}
\def\var{\mbox{var}}
\def\Tr{\mbox{Tr}}
\def\tm1{t\! - \! 1}
\def\tp1{t\! + \! 1}

\def\Xset{{\cal X}}

\newcommand{\one}{\mathbf{1}}
\newcommand{\abf}{\mathbf{a}}
\newcommand{\bbf}{\mathbf{b}}
\newcommand{\dbf}{\mathbf{d}}
\newcommand{\ebf}{\mathbf{e}}
\newcommand{\gbf}{\mathbf{g}}
\newcommand{\hbf}{\mathbf{h}}
\newcommand{\pbf}{\mathbf{p}}
\newcommand{\pbfhat}{\widehat{\mathbf{p}}}
\newcommand{\qbf}{\mathbf{q}}
\newcommand{\qbfhat}{\widehat{\mathbf{q}}}
\newcommand{\rbf}{\mathbf{r}}
\newcommand{\rbfhat}{\widehat{\mathbf{r}}}
\newcommand{\sbf}{\mathbf{s}}
\newcommand{\sbfhat}{\widehat{\mathbf{s}}}
\newcommand{\ubf}{\mathbf{u}}
\newcommand{\ubfhat}{\widehat{\mathbf{u}}}
\newcommand{\utildebf}{\tilde{\mathbf{u}}}
\newcommand{\vbf}{\mathbf{v}}
\newcommand{\vbfhat}{\widehat{\mathbf{v}}}
\newcommand{\wbf}{\mathbf{w}}
\newcommand{\wbfhat}{\widehat{\mathbf{w}}}
\newcommand{\xbf}{\mathbf{x}}
\newcommand{\xbfhat}{\widehat{\mathbf{x}}}
\newcommand{\xbfbar}{\overline{\mathbf{x}}}
\newcommand{\ybf}{\mathbf{y}}
\newcommand{\zbf}{\mathbf{z}}
\newcommand{\zbfbar}{\overline{\mathbf{z}}}
\newcommand{\zbfhat}{\widehat{\mathbf{z}}}
\newcommand{\Ahat}{\widehat{A}}
\newcommand{\Abf}{\mathbf{A}}
\newcommand{\Bbf}{\mathbf{B}}
\newcommand{\Cbf}{\mathbf{C}}
\newcommand{\Bbfhat}{\widehat{\mathbf{B}}}
\newcommand{\Dbf}{\mathbf{D}}
\newcommand{\Gbf}{\mathbf{G}}
\newcommand{\Hbf}{\mathbf{H}}
\newcommand{\Kbf}{\mathbf{K}}
\newcommand{\Pbf}{\mathbf{P}}
\newcommand{\Phat}{\widehat{P}}
\newcommand{\Qbf}{\mathbf{Q}}
\newcommand{\Rbf}{\mathbf{R}}
\newcommand{\Rhat}{\widehat{R}}
\newcommand{\Sbf}{\mathbf{S}}
\newcommand{\Ubf}{\mathbf{U}}
\newcommand{\Vbf}{\mathbf{V}}
\newcommand{\Wbf}{\mathbf{W}}
\newcommand{\Xhat}{\widehat{X}}
\newcommand{\Xbf}{\mathbf{X}}
\newcommand{\Ybf}{\mathbf{Y}}
\newcommand{\Zbf}{\mathbf{Z}}
\newcommand{\Zhat}{\widehat{Z}}
\newcommand{\Zbfhat}{\widehat{\mathbf{Z}}}
\def\alphabf{{\boldsymbol \alpha}}
\def\betabf{{\boldsymbol \beta}}
\def\mubf{{\boldsymbol \mu}}
\def\lambdabf{{\boldsymbol \lambda}}
\def\etabf{{\boldsymbol \eta}}
\def\xibf{{\boldsymbol \xi}}
\def\taubf{{\boldsymbol \tau}}
\def\sigmahat{{\widehat{\sigma}}}
\def\thetabf{{\bm{\theta}}}
\def\thetabfhat{{\widehat{\bm{\theta}}}}
\def\thetahat{{\widehat{\theta}}}
\def\mubar{\overline{\mu}}
\def\muavg{\mu}
\def\sigbf{\bm{\sigma}}
\def\etal{\emph{et al.}}
\def\Ggothic{\mathfrak{G}}
\def\Pset{{\mathcal P}}
\newcommand{\bigCond}[2]{\bigl({#1} \!\bigm\vert\! {#2} \bigr)}
\newcommand{\BigCond}[2]{\Bigl({#1} \!\Bigm\vert\! {#2} \Bigr)}

\def\Rect{\mathop{Rect}}
\def\sinc{\mathop{sinc}}
\def\Real{\mathrm{Re}}
\def\Imag{\mathrm{Im}}
\newcommand{\bkt}[1]{{\langle #1 \rangle}}



\begin{document}

\title{Problems:  Signal Space}
\author{Prof.\ Sundeep Rangan}
\date{}

\maketitle

\begin{enumerate}

\item \emph{Vector spaces and bases in $\F^N$.}  For each set $V$ below,
identify if it is a vector space or not.  If it is a vector space, find a basis.
If not, state the property that fails to occur.
\begin{enumerate}[(a)]
  \item $V = $ the set of $(x_1,x_2,x_3)$ such that $2x_1 + x_2 = 0$.
  \item $V = $ the set of $\xbf \in \R^3$ with $\|\xbf\|\leq 1$.
\end{enumerate}

\item \emph{Vector spaces of functions.}  For each set $V$ below,
state if $V$ is a subspace or not.  Explain.
\begin{enumerate}[(a)]
  \item Let $T> 0$ be some sampling period.
  $V = $ the set of $f(t)$ such that $f(nT)=0$ for all $n$.
  \item Let $f_{max} > 0$.  $V$ is the set of $s(t)$ that are bandlimited so that
  $S(f)=0$ for $|f| > f_{max}$.
  \item $V=$ set of functions on $[0,\infty)$ of the form, $f(t) = Ae^{-(t-\tau)}$ for 
  some $A$ and $\tau$.
  \item $V=$ set of functions on $[0,\infty)$ of the form, $f(t) = Ae^{-Bt}$ for some $A$ and $B$.
\end{enumerate}


\item \emph{Signal set and signal space}. Let $N$ and $K$ be constants and
consider the signal set ${\mathcal S}$ consisting
of signals $s[n]$ such that $s[n]=1$ in \emph{exactly} $K$ times $n\in [0,1,\ldots,N-1]$.
For all other $n$, $s[n]=0$.
\begin{enumerate}[(a)]
\item Find $M$, the number of signals in ${\mathcal S}$.
\item Find the number of degrees of freedom.
\item Find the rate of signal set.
\end{enumerate}
This type of signal set can encode information by the position of the non-zero elements.


\item \emph{Signal set and signal space.}  Consider the following four functions:
\[
    s(t) = e^{-At + B}, t \geq 0,
\]
where $A = 1$ or $2$ and $B=0$ or 1.
\begin{enumerate}[(a)]
\item Find a basis for a signal space containing the signal set.
Use a basis with a minimum number of signals.
\item Find the coordinates of each signals in the basis.
\end{enumerate}



\item \emph{Bandlimited channels.}  Suppose that a communication system is allocated
a channel 2.29 to 2.31 GHz and has 10\% overhead.
\begin{enumerate}[(a)]
\item What are the (complex) degrees of freedom per second?
\item What is the spectral efficiency required for 40 Mbps?
\item What is the rate if the system uses 16-QAM on every degree of freedom?
\item If the signal is received at -100 dBm, what is the average energy per degree of freedom.
\end{enumerate}



\item \emph{Orthonormal bases}  Suppose that a signal space has a basis $s_1(t)$, $s_2(t)$ with
\[
    \|s_1\|^2 = \|s_2\|^2 = 1, \bkt{s_1,s_2} = \rho.
\]
\begin{enumerate}[(a)]
\item Using Gram-Schmidt, find an orthonormal basis $u_1$, $u_2$ for the signal space.
\item Write $s_1$ and $s_2$ in terms of $u_1$ and $u_2$.
\item Suppose a signal is transmitted as,
\[
    s(t) = a_1 s_1(t) + a_2 s_2(t).
\]
Find the coordinates of $s(t)$ in the $u_1(t),u_2(t)$ basis.
That is, find $b_1,b_2$ in terms of $(a_1,a_2)$ such that $s(t) = b_1u_1(t) + b_2 u_2(t)$

\item Find $b_1,b_2$ for the four constellation points $a_1=a_2 = \pm 1$ and $\rho = 0.2$.
\end{enumerate}


\end{enumerate}

\end{document}

